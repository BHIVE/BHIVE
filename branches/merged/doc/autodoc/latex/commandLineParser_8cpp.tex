\section{commandLineParser.cpp File Reference}
\label{commandLineParser_8cpp}\index{commandLineParser.cpp@{commandLineParser.cpp}}


{\tt \#include \char`\"{}commandLineParser.hh\char`\"{}}\par
{\tt \#include $<$iostream$>$}\par
\subsection*{Functions}
\begin{CompactItemize}
\item 
bool {\bf parseArguments} (int argc, const char $\ast$argv[$\,$], map$<$ string, string $>$ \&argMap)
\begin{CompactList}\small\item\em Parses command line arguments from the console into an argMap. \item\end{CompactList}\item 
int {\bf parseAsInt} (map$<$ string, string $>$ \&argMap, string argName, int defaultValue)
\begin{CompactList}\small\item\em Looks up the argument in the argMap and tries to parse the value as an integer. \item\end{CompactList}\item 
double {\bf parseAsDouble} (map$<$ string, string $>$ \&argMap, string argName, double defaultValue)
\begin{CompactList}\small\item\em Looks up the argument in the argMap and tries to parse the value as a double. \item\end{CompactList}\end{CompactItemize}


\subsection{Function Documentation}
\index{commandLineParser.cpp@{commandLineParser.cpp}!parseArguments@{parseArguments}}
\index{parseArguments@{parseArguments}!commandLineParser.cpp@{commandLineParser.cpp}}
\subsubsection[parseArguments]{\setlength{\rightskip}{0pt plus 5cm}bool parseArguments (int {\em argc}, \/  const char $\ast$ {\em argv}[$\,$], \/  std::map$<$ std::string, std::string $>$ \& {\em argMap})}\label{commandLineParser_8cpp_db6c1353f5ffe64f1c40d49ddf1f4b74}


Parses command line arguments from the console into an argMap. 

Given the vector of strings taken from the command line, this function parses out all strings that start with a dash and identifies them as parameters, and attatches the parameter value to whatever string follows. For example, -file help.txt would add an entry to the argMap as a parameter named \char`\"{}file\char`\"{} with value \char`\"{}help.txt\char`\"{}. You can take a look at parseAsInt and parseAsDouble functions that can interpret the value as integers or doubles

\begin{Desc}
\item[Parameters:]
\begin{description}
\item[{\em argc}]the number of arguments \item[{\em argv}]the array of character arrays (strings) \item[{\em argMap}]the map that will be set when this function is called \end{description}
\end{Desc}
\begin{Desc}
\item[Returns:]bool true if successful, false if something went wrong \end{Desc}
\begin{Desc}
\item[Author:]Michael Sneddon \end{Desc}
\index{commandLineParser.cpp@{commandLineParser.cpp}!parseAsDouble@{parseAsDouble}}
\index{parseAsDouble@{parseAsDouble}!commandLineParser.cpp@{commandLineParser.cpp}}
\subsubsection[parseAsDouble]{\setlength{\rightskip}{0pt plus 5cm}double parseAsDouble (map$<$ string, string $>$ \& {\em argMap}, \/  string {\em argName}, \/  double {\em defaultValue})}\label{commandLineParser_8cpp_412e89bbcc308c634d3911a49280bbbc}


Looks up the argument in the argMap and tries to parse the value as a double. 

\begin{Desc}
\item[Parameters:]
\begin{description}
\item[{\em argMap}]the argMap to lookup, generally created by the parseArguments function \item[{\em string}]the name of the parameter to look up \item[{\em defaultValue}]the default value to return if the value was empty \end{description}
\end{Desc}
\begin{Desc}
\item[Returns:]int the parsed double if successful, otherwise the defaultValue given \end{Desc}
\begin{Desc}
\item[Author:]Michael Sneddon \end{Desc}


References Util::convertToDouble().\index{commandLineParser.cpp@{commandLineParser.cpp}!parseAsInt@{parseAsInt}}
\index{parseAsInt@{parseAsInt}!commandLineParser.cpp@{commandLineParser.cpp}}
\subsubsection[parseAsInt]{\setlength{\rightskip}{0pt plus 5cm}int parseAsInt (map$<$ string, string $>$ \& {\em argMap}, \/  string {\em argName}, \/  int {\em defaultValue})}\label{commandLineParser_8cpp_7e2b760fe8b56a663c12352037277dd0}


Looks up the argument in the argMap and tries to parse the value as an integer. 

\begin{Desc}
\item[Parameters:]
\begin{description}
\item[{\em argMap}]the argMap to lookup, generally created by the parseArguments function \item[{\em string}]the name of the parameter to look up \item[{\em defaultValue}]the default value to return if the value was empty \end{description}
\end{Desc}
\begin{Desc}
\item[Returns:]int the parsed int if successful, otherwise the defaultValue given \end{Desc}
\begin{Desc}
\item[Author:]Michael Sneddon \end{Desc}


References Util::convertToInt().